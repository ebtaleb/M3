\documentclass[11pt,a4paper,margin=0.5in]{report}

\usepackage[utf8]{inputenc}
\usepackage[francais]{babel}
\usepackage[margin=1in,footskip=0.25in]{geometry}
\usepackage{graphicx}
\usepackage[document]{ragged2e}
\usepackage{hyperref}

\title{Modular Mind Mapping - M3 \\ UPMC M2 STL TPDEV Projet}
\author{Elias Boutaleb}
\date{\today}

\begin{document}

\maketitle
\tableofcontents

\chapter{Introduction}

\section{Contexte}

\section{Fonctionnalités}

Un utilisateur a pour possibilité sur le site de :
%\begin{itemize}
%\end{itemize}

\chapter{Manuel utilisateur}

\section{Description de l'interface}

%\includegraphics[scale=0.33]{illus/main.png} \\[0.25in]

Voici la page principale de l'application après connexion.\\
La barre de navigation permet respectivement d'accéder à la page ci-dessus, de créer un nouvel évènement, de consulter son profil et de se déconnecter
de l'application.\\


\clearpage

\section{Cas d'utilisation}


\chapter{Architecture de l'application}

\section{Schéma}

%\begin{center}
%\includegraphics[scale=0.4]{illus/appschema.png} \\[0.25in]
%\end{center}

\section{Choix techniques}


\vspace{0.5cm}

module TimeEntry\footnote{\url{http://www.keith-wood.name/timeEntry.html}}.\\[0.1in]
La manipulation des cartes se fait avec l'aide de l'API Google Maps.

\chapter{Extensions et améliorations}

\section{Avantages de l'application}
\begin{enumerate}
    \item Elle est extensible, grâce à l'architecture MVC qui rend facile l'ajout de nouvelles pages et des routes URL correspondantes.
    %\begin{itemize}
    %\end{itemize}
    \item Elle via l'API REST.
\end{enumerate}

\section{Inconvénients de l'application}

\section{Améliorations à faire}

\section{Extensions possibles}
Les fonctionnalités suivantes pourraient être ajoutées à l'application:

\end{document}
